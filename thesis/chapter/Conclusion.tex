\chapter{Fazit} 

\section{Zusammenfassung}
Das Ziel dieser Bachelorarbeit war es, die Methoden von Image Colorization anhand Convolutional Neural Networks zu untersuchen, insbesondere
Klassifikationsmethoden. Durch die ausführliche Auseinandersetzung mit Klassifikationsmethoden und die Implementierung der gebrauchten Techniken
wie Binning, konnte eine Methode basierend auf den letzten Stand der Technik implementiert werden. Dieser Methode wurde angewandt und mit verschiedene
Datensätze getestet um die Ergebnisse zu untersuchen.
\\
Die Ergebnissen dieser Arbeit haben gezeigt dass die Methoden funktionieren und sogar mit wenigen Bildern und Epochen gute Ergebnissen produzieren.
Zahlreiche Experimente bestätigten dass Klassifikationsmethoden bessere Ergebnisse als Regressionsmethoden liefern. Diese Differenz ist deutlicher
bei Methoden die Optimierungstechniken, wie Class Rebalancing, anwenden.


\section{Kritischer Rückblick}
% TODO (Reflexion und Bewertung der Zielsetzung gegenüber erreichtem Ergebnis)
Der Fokus dieser Arbeit ist auf der Untersuchung der Methoden gelegt. Bei der Erstellung, Implementierung und Durchführung der Methode wurde
festgestellt dass das Problem von Image Colorization sehr komplex ist. Obwohl die erreichten Ergebnissen ausreichend für den Vergleich mit 
andere Methoden sind, wurde festgestellt dass ohne Optimierungstechniken eine Verbesserung der Qualität der Bilder sehr schwer ist.
\\
\\
Die Auswahl eines balancierten Datensatzes ist ein wichtiger Faktor für die Qualität der Ergebnissen und kann Overfitting verhindern. Die Große 
des Datensatzes ist auch entscheidend bei dem Training eines nicht vor-trainiertes Model.
\\
\\
Die Zielsetzung wurde erreicht in dem die Klassifikationsmethode bestätigt wurde, obwohl die Qualität der Ergebnissen durch weiteres Training
und Optimierung verbessert werden kann.

\section{Ausblick}
Die implementierte Methode dieser Arbeit hat viel Potential und kann durch Anpassung der Techniken und Anwendung von Optimierungstechniken
verbessert werden. Eine Anwendung auf andere und größere Datensätze wird mit Sicherheit bessere Ergebnissen produzieren. Es wäre auch interessant, 
die Methode auf Videos anzuwenden. 
\\
\\
Methoden von dem Bereich unüberwachtes Lernen werden heutzutage angewendet um das Image Colorization Problem zu lösen. Generative adversarial
Networks (GANs) werden in der Praxis öfters angewendet als normale \gls{CNN}s da dieser viel bessere Ergebnissen erzeugen.
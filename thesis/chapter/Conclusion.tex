\chapter{Fazit} 

\section{Zusammenfassung}
Das Ziel dieser Bachelorarbeit war es, die Methoden von Image Colorization anhand Convolutional Neural Networks zu untersuchen, insbesondere
Klassifikationsmethoden. Durch die ausführliche Auseinandersetzung mit Klassifikationsmethoden und die Implementierung der verwendeten Techniken
wie Binning, konnte eine Methode, basierend auf dem letzten Stand der Technik, implementiert werden. Diese Methode wurde angewandt und mit verschiedenen
Datensätzen getestet, um die Ergebnisse zu untersuchen.
\\
Die Ergebnisse dieser Arbeit haben gezeigt, dass die Methoden funktionieren und sogar mit weniger Bildern und Epochen gute Ergebnisse produzieren.
Zahlreiche Experimente bestätigten, dass Klassifikationsmethoden bessere Ergebnisse als Regressionsmethoden liefern. Diese Differenz ist deutlicher
bei Methoden die Optimierungstechniken, wie Class Rebalancing, anwenden.


\section{Kritischer Rückblick}
% TODO (Reflexion und Bewertung der Zielsetzung gegenüber erreichtem Ergebnis)
Der Fokus dieser Arbeit wurde auf die Untersuchung der Methoden gelegt. Bei der Erstellung, Implementierung und Durchführung der Methode wurde
festgestellt, dass das Problem von Image Colorization sehr komplex ist. Obwohl die erreichten Ergebnissen ausreichend für den Vergleich mit 
anderen Methoden sind, wurde festgestellt dass ohne Optimierungstechniken eine Verbesserung der Qualität der Bilder nur schwer zu erreichen ist.
\\
\\
Die Auswahl eines balancierten Datensatzes ist ein wichtiger Faktor für die Qualität der Ergebnisse und kann Overfitting verhindern. Die Größe 
des Datensatzes ist auch entscheidend bei dem Training eines nicht vor-trainierten Modells.
\\
\\
% TODO: what
Die Zielsetzung wurde erreicht in dem die Klassifikationsmethode bestätigt wurde, obwohl die Qualität der Ergebnisse durch weiteres Training
und Optimierung verbessert werden kann.

\section{Ausblick}
Die implementierte Methode dieser Arbeit hat viel Potential und kann durch Anpassung der Techniken und Anwendung von Optimierungstechniken
verbessert werden. Eine Anwendung auf andere und größere Datensätze wird mit Sicherheit bessere Ergebnisse produzieren. Die Methode auf Videos
anzuwenden würde ebenfalls interessante Ergebnisse liefern.
\\
\\
Methoden aus dem Bereich der unüberwachten Lernens werden heutzutage angewendet um das Image Colorization Problem zu lösen. Generative adversarial
Networks (GANs) werden in der Praxis öfter angewendet, als normale \gls{CNN}s da diese viel bessere Ergebnisse erzeugen.
\chapter{Konzeption}
% TODO: weiter schreiben
Dieses Kapitel beinhaltet alle Schritte für die Konzeption der angewandte Methoden. Außerdem wird der Datensatz und die verwendete Frameworks 
präsentiert.

\section{Image Colorization als Multimodales Problem}
Konventionelle automatische Methoden zielen darauf ab, die Farben für ein generiertes Bild so nah wie möglich an das Originale Bild vorherzusagen.
Diese Methoden verwenden ein MSE Loss der Vorhersagen die Weit von den Originalen Farbwerte entfernt liegen, stärker bestraft, als Farbwerte
die dichter an den Originalen Farbwerte liegen. Das führt, wie bei \ref{subsection:verwandte-arbeiten} beschrieben, zu entsättigte Bilder.
Die Gründe für diese Ergebnisse ist dass verschiedene Objekte, verschiedene Farben einnehmen können. Aus diesem Grund, behandelt die vorliegende
Arbeit das Problem als ein Multimodales Problem.

% TODO: finish this
\section{Farbraum}
Der Standard Farbraum der Bilder für die Methoden dieser Arbeit ist der RGB-Farbraum. Bei diesem Farbraum lässt sich schwer das Graustufen Bild 
von den Farbkanäle zu trennen, daher wird der Lab-Farbraum verwendet.
\\
\\
Bei den Lab-Farbraum können ohne Probleme die Farbkanäle ``ab'' von den Belichtungskanal ``L'' getrennt werden. Der Belichtungskanal ``L'' 
enthält das Graustufenbild die in den CNN eingespeist wird. Das generierte Bild wird für die Darstellung von dem Lab-Farbraum in den RGB-Farbraum 
konvertiert.

\begin{figure}[H]
  \vspace{1cm}
  \begin{subfigure}
    \centering
    \includegraphics[width=.5\textwidth]{resources/colorspace/image.jpg}
  \end{subfigure}
  \begin{subfigure}
    \centering
    \includegraphics[width=.5\textwidth]{resources/colorspace/grayscale.png}
  \end{subfigure}
  \begin{subfigure}
    \centering
    \includegraphics[width=.5\textwidth]{resources/colorspace/a_channel.png}
  \end{subfigure}
  \begin{subfigure}
    \centering
    \includegraphics[width=.5\textwidth]{resources/colorspace/b_channel.png}
  \end{subfigure}
  \caption{Originales Bild oben links, den Belichtungskanal ``L'' oben rechts, unten links den Farbkanal ``a'' und unten rechts den Farbkanal ``b''.}
  \label{fig:fig}
\end{figure}

% TODO: maybe new image
\section{Binning}
Binning ist eine Technik, die für die Bildverarbeitung verwendet wird. Binning wird, in dem Kontext von Image Colorization, als Eingruppierung 
von naheliegenden Farben definiert. Die Farben werden in gleich große Intervalle aufgeteilt. Diese Intervalle bezeichnet man im Englischen als
``\gls{bin}s''. Jedes dieser Intervalle wird durch einen \gls{bin} Index repräsentiert, somit reduziert sich die Anzahl der Klassen die vorhergesagt werden
können.

Als Beispiel für die Veranschaulichung wird der normalisierte Lab-Farbraum in 36 gleich große \gls{bin}s unterteilt. Da die Farbinformationen 
in den ``ab'' Farbkanäle kodiert sind, werden nur diese 2 Farbkanäle in \gls{bin}s klassifiziert. Auf dem Bild \ref{image:bins} ist der Farbkanal 
``a'' auf der x-Achse und der ``b'' Farbkanal auf der y-Achse abgebildet. Die Vierecke repräsentieren die \gls{bin}s. Die obere Zahl in den Bins 
symbolisiert den Index auf dem \gls{grid}, die untere Zahl symbolisiert den Bin Index. Der Index auf den \gls{grid} ist Bedeutsam für die Berechnung der Bins.

\begin{figure}[H]
  \centering
  \includegraphics[width=0.55\textwidth]{resources/bins/bins.jpg}
  \caption{
  \gls{grid} mit 36 bins. Die x-Achse bildet die Werte von dem Farbkanal ``a'' und die y-Achse die Werte von den Farbkanal ``b'' ab.
  }
  \label{image:bins}
\end{figure}

Für die Methoden dieser Arbeit wurden nur symmetrische \gls{grid}s verwendet, so ergibt zum Beispiel ein $6 \times 6$ Grid 36 Bins und 
ein $18 \times 18$ Grid 324 Bins.

\section{Netzwerkarchitektur}
Die Netzwerkarchitektur ist ein wichtiger Faktor dass u.a. die Ergebnisse beeinträchtigen kann. Wichtig um die Methoden zu vergleichen ist ein
leichtes Netz der wenige Parameter besitzt, schnell zu trainieren ist und gute Ergebnisse liefert.
\\
\\
Das Ziel von dem Netzwerk ist es, ein Graustufenbild als Input zu bekommen und eine Klasse (Bin) für jeden Pixel vorherzusagen. Das Output Volumen
hat die Dimensionen $ W_{Input} \times H_{Input} \times 1 $, wobei $W$ und $H$ die Breite und Höhe von dem Graustufenbild sind, mit einem Bin Index 
bei jedem Pixel. Dieser Ansatz ist auch bei Image Segmentation Probleme genutzt, wo ein Bild in das Netzwerk einspeist wird und als Output, wird 
ein Segmentation map, mit eine Klasse per Pixel, erzeugt. In der Regel hat jeder Klasse eine bestimmte Farbe und dadurch werden Objekte klassifiziert 
und getrennt. In dem Fall von Image Colorization bekommt jeder Pixel in dem Output Volumen eine Klasse (Bin) die nach der Vorhersage in 
einer Farbe umgewandelt wird.
\\
\\
Die Methode dieser Arbeit verwendet eine angepasste Version von einem U-Net Convolutional Neural Network \cite{ronneberger2015unet}. Ein U-Net
wird bei Image Segmentation angewendet und ist einer Art Autoencoder mit Skip Connections. Im vergleich zu Konventionelle Autoencoder, können
der Encoder und Decoder nicht getrennt voneinander verwendet werden und bei dem Decoder werden Transposed Convolutions, als Upsampling 
Methode, verwendet. Das U-Net verfügt außerdem über Skip Connections die ermöglichen fein-granuläre Details in dem Output Volumen zu 
wiederherstellen und helfen mit dem Vanishing Gradient Problem in Backpropagation. Skip Connections konkatenieren bestimmte Layers von dem
Encoder mit Layers von dem Decoder, mit der gleichen Dimensionen.
\\
\\
Für dieser Arbeit wurden 2 U-Nets mit verschiedene Größen verwendet. Ein U-Net für $ 32 \times 32 $ Bilder und ein U-Net für $ 128 \times 128 $ 
Bilder. Außerdem wurde das U-Net für $ 32 \times 32 $ Bilder angepasst damit es mit ein MSE Loss verwendet werden kann. Bei der Anpassung wurde
der Output Volumen zu $ W_{Input} \times H_{Input} \times 2 $ geändert, wobei das Netzwerk direkt die Werte für die ``ab'' Farbkanäle vorhersagt.
Bei der Verwendung von einem MSE Loss wird Binning nicht verwendet.

\section{Datensätze}
TODO
\section{Framework}
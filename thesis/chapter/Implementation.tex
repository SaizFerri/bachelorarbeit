\chapter{Implementierung}
% \label{cha:implmentation}
% Code Biespiel and Referenze: Code \ref{lst:tf_graph_save}. Nicht geeignet für große Code Biespiele. Verwenden Sie hierfür dne Anhang. Lediglich sehr relevante Kurzzeiler können auf diese Weise dargestellt werden. 
% \begin{lstlisting}[caption={Klasse Agent - Tensorflow Graph}, captionpos=b, label={lst:tf_graph_save}]
% from keras import backend as k
% ...
% def __init__(...):
% 	...
% 	self.graph = k.get_session().graph
% 	...
% \end{lstlisting}

TODO
\section{U-Net}
Für dieser Arbeit wurden 2 U-Nets mit verschiedene Größen verwendet. Ein U-Net für $ 32 \times 32 $ Bilder und ein U-Net für $ 128 \times 128 $ 
Bilder. Außerdem wurde das U-Net für $ 32 \times 32 $ Bilder angepasst damit es mit ein MSE Loss verwendet werden kann. Bei der Anpassung wurde
der Output Volumen zu $ W_{Input} \times H_{Input} \times 2 $ geändert, wobei das Netzwerk direkt die Werte für die ``ab'' Farbkanäle vorhersagt.
Bei der Verwendung von einem MSE Loss wird Binning nicht verwendet.
% TODO

% \section{Beispiel Unterkapitel}
% TODO
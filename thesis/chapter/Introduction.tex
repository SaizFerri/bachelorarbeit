\chapter{Einleitung}
Das Kapitel Einleitung verschafft einen Überblick über die Motivation, die angestrebte Zielsetzung, das
genaue Vorgehen und den Aufbau der Arbeit.

% TODO: weiter machen
\section{Motivation}
Jeder hat sich sicherlich gefragt, vor allem wenn es um Familienbilder geht, wie
ein Graustufenbild in Farbe aussehen würde. Es wäre faszinierend zu sehen wie die Welt von damals
in Farbe aussehen würde, etwas dass mich seit jeher fasziniert hat. Ein Graustufenbild kann
interaktiv von einem Mensch gefärbt werden, sodass Farben möglichst akkurat
vergeben werden. Wenn jedoch mehrere Tausend Bilder zu bearbeiten sind, wird das einige
Zeit in Anspruch nehmen. Dieses Problem kann mit Deep Learning gelöst werden, indem
ein Algorithmus selbstständig und möglichst realistisch, ein Graustufenbild
färbt. Der Prozess der Einfärbung eines Bildes ist ein aktives Forschungsgebiet
in Deep Learning. Es gibt bereits Methoden, basierend auf Convolutional
Neuronal Networks, die sehr realistische Ergebnisse liefern.

\section{Zielsetzung}
Das Ziel dieser Arbeit ist die bestehenden Methoden zu untersuchen, zu implementieren und zu vergleichen. Der Fokus wird
auf Methoden, die das Problem von Image Colorization als ein Multimodales Problem behandeln und implementiert auf Grund dessen, 
Klassifikationsmethoden.

\section{Vorgehensweise und Aufbau der Arbeit}
Die vorliegende Arbeit lässt sich in fünf Hauptkapitel aufteilen. Zu Beginn wird eine ausführliche Erläuterung der Grundlagen gegeben,
um die Methoden und Techniken der Arbeit zu verstehen. Anschließend werden die Konzepte, Techniken und Methoden präsentiert.
Um auf die bevorzugte Methode aufbauen zu können werden einige Datensätze gebraucht, diese werden bei den Konzepten präsentiert.
Nachdem auch diese erläutert wurden, wird die Implementierung erklärt. Darauf folgend werden zahlreiche Experimente mit verschiedenen
Hyperparametern durchgeführt. Abschließend werden die Experimente evaluiert und mit anderen Methoden verglichen.
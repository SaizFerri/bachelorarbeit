\chapter{Einleitung}
Das Kapitel Einleitung verschafft einen

\section{Motivation}
Jeder hat sich sicherlich gefragt, vor allem wenn es um Familienbilder geht, wie
würde ein schwarz/weiß Bild in Farbe aussehen. Ein schwarz/weiß Bild kann
interaktiv von einem Mensch gefärbt werden, so dass Farben möglichst akkurat
vergeben werden. Wenn man aber mehrere Tausend Bilder hat, wird das einige
Zeit dauern. Dieses Problem kann mit Deep Learning gelöst werden, indem
ein Algorithmus selbstständig und möglichst realistisch, ein schwarz/weiß Bild
färbt. Der Prozess der Einfärbung eines Bildes ist ein aktives Forschungsgebiet
in Deep Learning. Es gibt bereits Methoden, basierend auf Convolutional
Neuronal Networks, die sehr realistische Ergebnisse liefern.

\section{Zielsetzung}
Das Ziel dieser Arbeit ist die bestehende Methoden zu untersuchen, zu implementieren und zu vergleichen. Diese Arbeit setzt den Fokus
auf Methoden die das Problem von Image Colorization als ein Multimodales Problem behandeln und implementiert auf Grund dessen, 
Klassifikationsmethoden.

\section{Vorgehensweise und Aufbau der Arbeit}
Die vorliegende Arbeit lässt sich in fünf Hauptkapiteln aufteilen. Zu Beginn wird eine ausführliche Erläuterung der Grundlagen gegeben,
um die Methoden und Techniken der Arbeit zu verstehen. Anschließend werden die Konzepte, Techniken und bevorzugte Methode präsentiert.
Um auf die Methode aufbauen zu können werden einige Datensätze gebraucht, diese werden auch unter Konzepte präsentiert.
Nach dem die Konzepte präsentiert wurden, wird die Implementierung erklärt. Darauf folgend werden Zahlreiche Experimente mit verschiedene
Hyperparameter durchgeführt. Abschließend werden die Experimente evaluiert und mit andere Methoden verglichen.